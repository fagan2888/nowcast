\documentclass[12pt]{article}
\usepackage[
	a4paper,
	left   = 2.5cm,
	right  = 2.5cm,
	top    = 2.0cm,
	bottom = 2.0cm
	]{geometry}
\usepackage{fancyhdr}
\usepackage{amsmath,amsthm,amssymb, bbm}
\usepackage{graphicx,tikz,pgfplots}
\usepackage{hyperref}
\usepackage{lipsum}
\usepackage{placeins}
\usepackage{caption}
\usepackage{subcaption}
\usepackage[title, titletoc, toc]{appendix}
\usepackage[sort, comma]{natbib}
\usepackage{setspace}
\usepackage{pdflscape}
\onehalfspacing


\newcommand{\distas}[1]{\mathbin{\overset{#1}{\kern\z@\sim}}}%

%-----------------------------------------------------%
\title{Nowcasting Report of the US Economy}
\author{
	Macrosynergy Partners LLP
	}

\date{\today}

%-----------------------------------------------------%
\begin{document}
\maketitle

\begin{abstract}
	Report of the Nowcasting model at \emph{Macrosynergy Partners LLP} that summarise the forecasts of the US Economy as of \today. Furthermore the appendix gives a (currently incomplete) summary of the dynamic factor model used to generate these forecasts.
\end{abstract}

%\newpage
%\tableofcontents
%\newpage

%----------------------------%
%% --    Introduction    -- %%
%----------------------------%

%------------------------------------------------------%
\section{Current Status of the Nowcasting Project}
\begin{enumerate}
	\item	The difference between a VW and a Rolls-Royce, the current model is in the VW category
	\item 	In the key report we summarise the key predictions from the model based on the joint co-movement of the variables (and thereby the state of the US economy) as well as an explicit modelling of the idiosyncratic component for each of the series.
\end{enumerate}




%--------------------------------------------------------------------%

\section{Main Forecasts}

\begin{figure}[ht]
	\centering
	\caption{Key Now-Casts and Forecasts}
	%\includegraphics[scale=1]{graphs/KeyVariables.pdf}
	\caption*{\small{Datasource: Macrobond Financial AB}}
\end{figure}


\begin{landscape}
	\begin{table}[ht]
		\caption{United States: Nowcasting Model output}
		\begin{tabular}{{ll}*{1}{|}*{3}{c}*{1}{|}*{3}{c}*{1}{|}*{1}{ll}}
	\hline
	&&\multicolumn{3}{c|}{2016 Q4} & \multicolumn{3}{c|}{2017 Q1} & & \\
	 & & 2016M10& 2016M11& 2016M12& 2017M1& 2017M2& 2017M3 & Units & Next Event \\ \hline\hline
	\multicolumn{2}{l|}{\textbf{Surveys}} & && & && && \\

	\phantom{a} &Manufacturing, PMI& 51.9& 53.2& 54.7& \textcolor{green}{54.0}& \textcolor{green}{53.79}& \textcolor{green}{54.64}& Index & 2017-2-1 \\ 

	\phantom{a} &Non-Manufacturing, PMI& 54.8& 57.2& 57.2& \textcolor{green}{57.17}& \textcolor{green}{54.86}& \textcolor{green}{57.25}& Index & 2017-2-3 \\ 

	\phantom{a} &Consumer Sentiment Index& 87.2& 93.8& 98.2& \textcolor{green}{95.2}& \textcolor{green}{89.48}& \textcolor{green}{95.15}& Index & 2017-1-13 \\ 

	\phantom{a} &Business Outlook Survey, Manufacturing& 9.7& 7.6& 21.5& \textcolor{green}{17.47}& \textcolor{green}{12.25}& \textcolor{green}{9.11}& Index & 2017-1-12 \\ 

	\phantom{a} &Business Surveys, ISM Chicago& 50.6& 57.6& 54.6& \textcolor{green}{54.58}& \textcolor{green}{50.91}& \textcolor{green}{57.87}& Index & 2017-1-31 \\ 

	\phantom{a} &Consumer Sentiment Index& 100.8& 109.4& 113.7& \textcolor{green}{109.99}& \textcolor{green}{104.87}& \textcolor{green}{112.04}& Index & 2017-1-31 \\ 

	\multicolumn{2}{l|}{\textbf{Production and Trade}} & && & && && \\

	\phantom{a} &Retail Sales& 6.67& 7.01& \textcolor{red}{7.42}& \textcolor{green}{5.83}& \textcolor{green}{5.09}& \textcolor{green}{4.74}& QoQ, AR ($\%$) & 2017-1-13 \\ 

	\phantom{a} &Capacity Utilization& 75.38& 75.0& \textcolor{red}{75.16}& \textcolor{green}{75.12}& \textcolor{green}{75.23}& \textcolor{green}{74.9}& $\%$ & 2017-1-18 \\ 

	\phantom{a} &Industrial Production& -0.85& -2.28& \textcolor{red}{0.81}& \textcolor{green}{1.49}& \textcolor{green}{1.84}& \textcolor{green}{2.02}& QoQ, AR ($\%$) & 2017-1-18 \\ 

	\phantom{a} &Construction Started, Residential& 38.18& -26.27& \textcolor{red}{-4.94}& \textcolor{green}{0.61}& \textcolor{green}{1.97}& \textcolor{green}{2.22}& QoQ, AR ($\%$) & 2017-1-19 \\ 

	\phantom{a} &Construction by Status& 38.63& 20.31& \textcolor{red}{13.19}& \textcolor{green}{8.67}& \textcolor{green}{6.04}& \textcolor{green}{4.49}& QoQ, AR ($\%$) & 2017-1-19 \\ 

	\phantom{a} &Unfilled Orders, Durable Goods& 1.63& 1.66& \textcolor{red}{4.93}& \textcolor{green}{4.8}& \textcolor{green}{4.68}& \textcolor{green}{4.57}& QoQ, AR ($\%$) & 2017-1-27 \\ 

	\phantom{a} &New Orders, Durable Goods& 21.58& 2.24& \textcolor{red}{4.05}& \textcolor{green}{3.87}& \textcolor{green}{3.74}& \textcolor{green}{3.62}& QoQ, AR ($\%$) & 2017-1-27 \\ 

	\phantom{a} &Manufacturers' Inventories& 0.21& 0.63& \textcolor{red}{2.35}& \textcolor{green}{2.31}& \textcolor{green}{2.27}& \textcolor{green}{2.23}& QoQ, AR ($\%$) & 2017-1-27 \\ 

	\multicolumn{2}{l|}{\textbf{Labour Market}} & && & && && \\

	\phantom{a} &Hours Worked, Average Weekly& -1.19& 0.0& 0.0& \textcolor{green}{-0.01}& \textcolor{green}{-0.01}& \textcolor{green}{-0.02}& QoQ, AR ($\%$) & 2017-2-3 \\ 

	\phantom{a} &Unemployment& 4.8& 4.6& 4.7& \textcolor{green}{4.75}& \textcolor{green}{4.7}& \textcolor{green}{4.52}& $\%$ & 2017-2-3 \\ 

	\phantom{a} &Employment, Nonfarm, Payroll& 1.43& 1.51& 1.36& \textcolor{green}{2.33}& \textcolor{green}{2.2}& \textcolor{green}{2.08}& QoQ, AR ($\%$) & 2017-2-3 \\ 

	\multicolumn{2}{l|}{\textbf{Consumption and Income}} & && & && && \\

	\phantom{a} &Consumer Price Index& 1.63& 1.68& \textcolor{red}{1.49}& \textcolor{green}{1.76}& \textcolor{green}{1.84}& \textcolor{green}{1.83}& Annual rate ($\%$) & 2017-1-18 \\ 

	\phantom{a} &Personal Consumption Expenditures& 2.13& 3.1& \textcolor{red}{4.64}& \textcolor{green}{3.92}& \textcolor{green}{3.5}& \textcolor{green}{3.25}& QoQ, AR (\%) & 2017-1-30 \\ 

	\multicolumn{2}{l|}{\textbf{Quarterly Series}} & && & && && \\

	\phantom{a} &Real Gross Domestic Product&& & \textcolor{red}{3.18}&& & \textcolor{green}{2.78}& QoQ, SAAR (\%) & 2017-1-27 \\ 
 \hline
\end{tabular}

		\label{tab:table4}
		\centering
		\caption*{\small{Red numbers indicates a Nowcast, green numbers are forecasts. Datasource: Macrobond Financial AB}}
	\end{table}
\end{landscape}



\begin{figure}[ht]
	\centering
	\caption{Estimated Common Factors}
	%\includegraphics{graphs/estimatedFactors.pdf}
	\caption*{\small{Datasource: Macrobond Financial AB}}
\end{figure}
\FloatBarrier
%%%%%%%%%%%%%%%%%%%%%%%%%%%%%%%%%%%%%%
\section{Investigating US Retail Sales}


\begin{figure}[ht]
	\centering
	\caption{Retail Sales}
	%\includegraphics{graphs/retailSalesInvestigated.pdf}
	\caption*{\small{Datasource: Macrobond Financial AB}}
\end{figure}


\begin{table}[ht]
	\caption{United States: Retail Sales}
	\begin{tabular}{l || c|ccc|c || cc}
 Date & $\mu$ &$x_{t}$ & $\lambda$ & $ \hat{\psi}_{t} = \lambda \hat{x}_{t}$& $\hat{u}_{t}$ & $y_{t}^{f} = \mu + \hat{\psi}_{t} + \hat{u}_{t} $ & $y_{t} = y_{t}^{f}+\varepsilon_{t}$\\
 \hline\hline	2016-10 & 4.15& 0.14 & 2.27 & 0.31& 1.83 & 6.3 & 6.67 \\
	2016-11 & 4.15& 0.77 & 2.27 & 1.74& 2.79 & 8.68 & 7.01 \\
	2016-12 & 4.15& 0.49 & 2.27 & 1.11& 2.57 & 7.82 & NA \\
	2017-1 & 4.15& 0.43 & 2.27 & 0.98& 1.14 & 6.27 & NA \\
	2017-2 & 4.15& 0.39 & 2.27 & 0.87& 0.51 & 5.53 & NA \\
	2017-3 & 4.15& 0.34 & 2.27 & 0.78& 0.23 & 5.15 & NA \\
	2017-4 & 4.15& 0.3 & 2.27 & 0.69& 0.1 & 4.94 & NA \\
	2017-5 & 4.15& 0.27 & 2.27 & 0.61& 0.04 & 4.8 & NA \\
	2017-6 & 4.15& 0.24 & 2.27 & 0.54& 0.02 & 4.71 & NA \\
	2017-7 & 4.15& 0.21 & 2.27 & 0.48& 0.01 & 4.64 & NA \\
	2017-8 & 4.15& 0.19 & 2.27 & 0.43& 0.0 & 4.58 & NA \\
	2017-9 & 4.15& 0.17 & 2.27 & 0.38& 0.0 & 4.53 & NA \\
 \hline
\end{tabular}

	\label{tab:tableRetail}
	\centering
	\caption*{\small{Datasource: Macrobond Financial AB}}
	%\caption*{\small{Red numbers indicates a Nowcast, green numbers are forecasts}}
\end{table}

%-----------------------------------%%
\FloatBarrier
\section{The individual fit of the series}



\begin{figure}[ht]
	\centering
	\caption{Surveys}
	%\includegraphics{graphs/Surveys.pdf}
	\caption*{\small{Datasource: Macrobond Financial AB}}
\end{figure}



\begin{figure}[ht]
	\centering
	\caption{Production and Trade}
	%\includegraphics{graphs/Production and Trade.pdf}
	\caption*{\small{Datasource: Macrobond Financial AB}}
\end{figure}


\begin{figure}[ht]
	\centering
	\caption{Labour Market}
	%\includegraphics{graphs/Labour Market.pdf}
	\caption*{\small{Datasource: Macrobond Financial AB}}
\end{figure}


\begin{figure}[ht]
	\centering
	\caption{Consumption and Income}
	%\includegraphics{graphs/Consumption and Income.pdf}
	\caption*{\small{Datasource: Macrobond Financial AB}}
\end{figure}

\FloatBarrier


\section{Extensions}
The next steps for the model could be some of the following.

\subsection{Work in progress}
\begin{enumerate}
	\item	Robustness: Historical performance of the model and benchmarking against alternative forecasters and models
	\begin{enumerate}
		\item 	Survey of Professional Forecasters (SPF) from Philidelphia Fed, NY Fed Nowcast and Atlanta Fed GDPnow
		\item 	Various Econometric models (such as the constant growth model: $y_{t} = \mu + u_{t}$, AR process: $y_{t} = \mu + \rho y_{t-1} + u_{t}$, Local model: $y_{t} = \mu_{t} + u_{t}, \quad \mu_{t}=\mu_{t-1}+\epsilon_{t}$)
	\end{enumerate}
	\item 	Decompose the forecast into the predictable component (common component and predictable idiosyncratic component) and non-predictable
	\item 	Scenario based forecasting: Condition on the forecast of the next release for a series being higher or lower than the mean forecast of it ($\pm 2 st.d.$), how will that shift the forecast for our key variables?
	\item 	Decompose the releases into news and noise, and weight the innovation of the other series in terms of deviation from forecast (similar to what nowcasting.com is doing)
\end{enumerate}

\subsection{The next step?}
\begin{enumerate}
	\item	Improve on the estimation methodology (Maximum likelihood or Bayesian estimation)
	\item	Model the trend and Stochastic volatility for the series
	\item 	KloFlow v2: Nowcasting flow of funds?
	\item 	All the infrastructure is scalable for developed economies and (should be) possible to implement for emerging economies
\end{enumerate}

%--------------------------------------------------%


\subsection{Transformations of the data}

\begin{table}[ht]
	\centering
	\caption{Transformation of the data}
	\begin{tabular}{lcc}
		\hline\\
		Code & Model Transformation & Final Transformation\\ \hline\hline
		0 	& $x_{t}$ & $x_{t}$\\
		1 	& $\log(x_{t})$ & $x_{t}$\\
		2	& $\Delta_{3} x_{t}$ & $x_{t}$\\
		3 	& $400\Delta_{3} \log\left( x_{t} \right)$ &  $400\Delta_{3} \log\left( x_{t} \right)$\\
		4 	& $100\Delta_{3}\left(\Delta_{12}\log(x_{t})\right)$ & $100\Delta_{12}\log(x_{t})$\\ \hline
	\end{tabular}
\end{table}

\FloatBarrier
%--------------------------%


%------------------------------------%
\bibliography{mybib}{}
\bibliographystyle{plainnat}
%------------------------------------%


%---------------------------------------------------------------------%
\begin{appendices}

%% --- Aggregation Methodology -- %%

%---------------------------------%
%% -- Description of the model -- %%
%---------------------------------%

\section{Description of the model}

Following \cite{AntolinDiazDrechselPetrella2016} we model the observation (measurement) equation using the assumption of all monthly variables of the quarter are available:

\begin{equation}
y_{t}^{+} = \mu + \Lambda f_{t} + u_{t}
\end{equation}

The state-equation of the common factors are given by a VAR($p_{f}$) model:
\begin{equation}
f_{t} = \phi_{1} f_{t-1} +\phi_{2} f_{t-2} +  \dots + \phi_{p} f_{t-p_{f}}  + \epsilon_{t}, \quad \epsilon_{t}\sim N(0, Q).
\end{equation}

Or re-written:

\begin{equation}
\begin{aligned}
\left(I_{n} - \Theta(L) \right) u_{t}	&= \varepsilon_{t}, 	\quad 	&&\varepsilon_{t}\sim N(0,H)\\
\left(I_{r} - \phi(L)  \right) f_{t} 	&= \eta_{t}, 			\quad 	&&\eta_{t}\sim N(0,Q).
\end{aligned}
\end{equation}

The factor loadings ($\Lambda$) are not time-invariant and the idiosyncratic components $u_{t}$ follows an $ma(p_{u})$ process. 

Following \cite{MarianoMurasawa2003}, we approximate the quarterly variables as unobserved variables with the measurement equation:

\begin{equation}
\underset{n_{q}.1}{y_{t}^{Q}} - \mu^{Q} = \lambda^{Q} \widetilde{f}_{t} + \widetilde{u}_{t}^{Q}
\end{equation}

With the variables given as:
\begin{equation*}
\begin{aligned}
\widetilde{f}_{t} &= \frac{1}{3} f_{t} + \frac{2}{3} f_{t-1} + f_{t-2} + \frac{2}{3} f_{t-3} + \frac{1}{3} f_{t-4}\\
\widetilde{u}_{t}^{Q} &= \frac{1}{3} u_{t}^{Q} + \frac{2}{3} u_{t-1}^{Q} + u_{t-2}^{Q} + \frac{2}{3} u_{t-3}^{Q} + \frac{1}{3} u_{t-4}^{Q}.
\end{aligned}
\end{equation*}

%And the idiosyncratic component of the quarterly variables are given by:
%\begin{equation}
%	(I_{n_{q}} - \Theta^{(q)}(L)) u_{t}^{(q)} =  \varepsilon_{t}^{(q)}
%\end{equation}

Following \cite{DurbinKoopman2012} and \cite{Harvey1989} and inverting the idiosyncratic component we can re-write the model in terms of quasi-differences for the monthly variables:

\begin{equation}
\begin{aligned}
\widetilde{y}_{t} 	&= C X_{t} + \tilde{\epsilon}_{t},	\quad &\tilde{\epsilon}_{t} 		&\sim N(0, \widetilde{H}),\\
X_{t} 			&= A X_{t-1} + \tilde{\eta}_{t},		\quad &\tilde{\eta}_{t} 	&\sim N(0, \widetilde{Q}).
\end{aligned}
\end{equation}

Where we have written out the monthly observations in terms of quasi-differences (again following \cite{AntolinDiazDrechselPetrella2016}):

\begin{equation*}
\underset{n_{y}.1}{\widetilde{y}_{t}} =
\begin{bmatrix}
\left(I_{n_{m}} - \Theta^{(m)}(L)\right)\left(y_{t}^{(m)} - \mu^{(m)}\right)\\
y_{t}^{(q)} - \mu^{(q)}
\end{bmatrix}.
\end{equation*}

The state-variable ($X_{t}$) is found by first defining the two auxiliary variables: $pp =\max(p_{f}, p_{u}+1, 5)$ and $qq=\max(p_{u}, 5)$, where $p_{f}$ and $p_{u}$ is the lags for the dynamic process of the common components and the idiosyncratic components.
Then noting that the dimension of $X_{t}$ is $n_{x} = pp \times r + qq \times n_{q}$ we can set up the state-vector as:

\begin{equation*}
\underset{n_{x}.1}{X_{t}'} =
\begin{bmatrix}
\underset{r.1}{f_{t}}' 				&
\underset{r.1}{f_{t-1}}'			&
\dots								&
\underset{r.1}{f_{t-pp + 1}}'		&
\underset{n_{q}.1}{u_{t}^{(q)}}'	&
\dots								&
\underset{n_{q}.1}{u_{t-qq+1}^{(q)}}'
\end{bmatrix}
\end{equation*}

The measurement equation matrix is given by and the variance-covariance of the idiosyncratic component are given by:

\begin{equation*}
\underset{n_{y}.n_{x}}{C(\Lambda)} =
\begin{bmatrix}
% Monthly Variables Loading of states
\underset{n_{m}.pp\times r}{c_{m}} 
& \underset{n_{m}.(qq\times n_{q})}{\textbf{0}}\\
% Quarterly Variable Loading of states
\underset{n_{q}.r}{\lambda^{q}} \times \underset{r.pp\times r}{ c_{Q}^{f} } 
& \underset{n_{q}.(qq\times n_{q})}{ c_{Q}^{u} }
\end{bmatrix}
\end{equation*}

The loadings are determined by the matrices:

\begin{equation}
\begin{aligned}
\underset{r.(p_{f}-1)r}{ c_{Q}^{f} } &= 
\begin{bmatrix}
\frac{1}{3} 
& \frac{2}{3}
& 1
& \frac{2}{3}
& \frac{1}{3}
& \underset{1.(pp-5)}{\textbf{0}} 
\end{bmatrix}
\otimes I_{r}\\
\underset{n_{q}.qq\times n_{q}}{ c_{Q}^{u} } &=
\begin{bmatrix}
\frac{1}{3}
& \frac{2}{3}
& 1
& \frac{2}{3}
& \frac{1}{3}
& \underset{1.(qq-5)}{\textbf{0}}
\end{bmatrix}
\otimes I_{n_{q}}
\end{aligned}
\end{equation}

and for the monthly Variables:
\begin{equation}
\underset{n_{m}.pp\times r}{c_{m}(\Theta^{m}, \lambda_{m})} = 
\begin{bmatrix}
\begin{bmatrix}
I_{n_{m}} & \underset{n_{m}.p_{u} \times n_{m}}{-\Theta^{m}} 
\end{bmatrix}
\times \left(I_{(p_{u}+1)} \otimes \underset{n_{m}.r}{\lambda^{m}}\right) & \underset{n_{m}.(pp-p_{u}-1)r}{\textbf{0}}
\end{bmatrix}
\end{equation}

Where the lag coeffiicents are given by (all diagonal matrices):

\begin{equation*}
\Theta_{m} =
\begin{bmatrix}
%\theta_{1}^{m} & \theta_{2}^{m} & \dots & \theta_{p_{u}}^{m}
diagm(\theta_{1}^{m}) & diagm(\theta_{2}^{m}) & \dots & diagm(\theta_{p_{u}}^{m})
\end{bmatrix}
\end{equation*}


The idiosyncratic component and it's variance covariance in the measurement equation is defined as:
\begin{equation*}
\tilde{\varepsilon}_{t}' = 
\begin{bmatrix}
\underset{n_{m}.1}{\varepsilon^{(m)'}_{t}} &
\underset{n_{q}.1}{\textbf{0}'}
\end{bmatrix},
\quad
\widetilde{H} = 
\begin{bmatrix}
\underset{n_{m}.n_{m}}{H_{m}} 	&  \underset{n_{m}.n_{q}}{\textbf{0}} \\
\underset{n_{q}.n_{m}}{\textbf{0}} 	& \underset{n_{q}.n_{q}}{\textbf{0}}
\end{bmatrix}
\end{equation*}

where the variance-covariance matrix is a diagonal matrix.
This could be exploited to reduce the number of parameters of the model.


The system matrix is given by:

\begin{equation}
\underset{n_{x}.n_{x}}{A(\Phi, \Theta^{Q})} = 
\begin{bmatrix}
%% Common factors
\underset{pp \times r. pp \times r}{a_{f}(\Phi)} 
& \underset{pp \times r. qq \times n_{q}}{\textbf{0}}\\
%% Quarterly Idiosyncratic Component
\underset{qq \times n_{q}.pp \times r}{\textbf{0} }	  
& \underset{qq\times n_{q}.qq \times n_{q}}{a_{Q} \left( \Theta^{Q} \right)}
\end{bmatrix}
\end{equation}

With the individual elements given by:

\begin{equation*}
\underset{pp\times r.pp \times r}{a_{f}} =
\begin{bmatrix}
\underset{r.p_{f}\times r}{\Phi} &
\underset{r.(pp-p_{f})r}{\textbf{0}} \\
%%
I_{(pp - 1)r} &
\underset{(pp - 1)r.r}{\textbf{0}}
\end{bmatrix}
, \quad
\underset{r.p_{f}r}{\Phi} =
\begin{bmatrix}
\underset{r.r}{\phi_{1}} 
& \underset{r.r}{\phi_{2}} 
& \dots 
& \underset{r.r}{\phi_{p_{f}} }
\end{bmatrix}
\end{equation*}


The quarterly idiosyncratic component:

\begin{equation*}
\underset{qq \times n_{q}. qq \times n_{q}}{a_{Q}} =
\begin{bmatrix}
\underset{n_{q}.p_{u}\times n_{q}}{\Theta_{Q}} 	&
\underset{n_{q}.(qq - p_{u}) \times n_{q}}{\textbf{0}}\\
%%
I_{(qq - 1) n_{q}}  & 
\underset{(qq - 1) n_{q} . n_{q}}{\textbf{0}}\\
\end{bmatrix}
\end{equation*}


And lastly the state innovations are given by:
\begin{equation*}
\underset{n_{x}.1}{\tilde{\eta}_{t}}' = 
\begin{bmatrix}
\underset{r.1}{\eta_{t}}' & 
\underset{(pp - 1)r.1}{\textbf{0}}' & 
\underset{n_{q}.1}{\varepsilon_{t}^{Q}}' & 
\underset{(qq -1) n_{q}.1}{\textbf{0}}'
\end{bmatrix},
\end{equation*}

And the variance-covariance matrix of the state are given by:

\begin{equation*}
\underset{n_{x}.n_{x}}{ \widetilde{Q} } =
\begin{bmatrix}
%% first row
\underset{r.r}{Q} 
& \underset{r.(pp-1)r}{\textbf{0}}
& \underset{r.n_{q}}{\textbf{0}}
& \underset{r.(qq-1)n_{q}}{\textbf{0}}\\
%% second row
\underset{(pp-1)r.r}{\textbf{0}}
& \underset{(pp-1)r.(pp-1)r}{\textbf{0}}
& \underset{(pp-1)r.n_{q}}{\textbf{0}}
& \underset{(pp-1)r.(qq-1)n_{q}}{\textbf{0}}  \\
%% third row
\underset{n_{q}.r}{\textbf{0}} 
& \underset{n_{q}.(pp-1)r}{\textbf{0}}
& \underset{n_{q}.n_{q}}{ H_{Q} }
& \underset{n_{q}.(qq-1)n_{q}}{\textbf{0}} \\
%% fourth row
\underset{(qq-1)n_{q}.r}{\textbf{0}}	
& \underset{(qq-1)n_{q}.(pp-1)r}{\textbf{0}}
& \underset{(qq-1)n_{q}.n_{q}}{\textbf{0}}
& \underset{(qq-1)n_{q}.(qq-1)n_{q}}{\textbf{0}}
\end{bmatrix}
\end{equation*}

We check whether the system is observable and controllable following the rank conditions of \citet[pp.113--117]{Harvey1989}.



%---------------------------------------------------------------------%
\section{Kalman filter, Kalman Smoother and the log--likelihood}

Firstly following \cite{Harvey1989} we check whether the state-space of the model is controllable and observable. 

Add the following derivations and use of:
\begin{enumerate}
	\item 	Kalman filter
	\item 	Kalman smoother
	\item 	log--likelihood
\end{enumerate}


First, recall the system:

\begin{equation}
\begin{aligned}
\widetilde{y}_{t} 			&= C X_{t} + \tilde{\epsilon}_{t}, 	\quad &\epsilon_{t} 	        &\sim N(0, \widetilde{H} ),\\
X_{t} 			&= A X_{t-1} + \tilde{\eta}_{t}		\quad &\tilde{\eta}_{t} 		&\sim N(0, \widetilde{Q}).
\end{aligned}
\end{equation}


\subsection{The Kalman filter}

The following is the algorithm for filtering out (recursively) the unobserved state.


First we define the forecast error of the model as (conditional of the parameters for iteration $i$):
\begin{equation}
\widehat{v}_{t}^{(i)} = \widetilde{y}_{t} - \widehat{C}^{(i)} \widehat{X}_{t|t-1}^{(i)}
\end{equation}

The equations for updating the Kalman filter are given by (need to add missing observation adjustment!):

\begin{equation*}
\begin{aligned}
\underset{n_{y}.1}{\widehat{v}_{t}^{(i)}} 			&= \widetilde{y}_{t} - \widehat{C}^{(i)} \widehat{X}_{t|t-1}^{(i)}\\
\underset{n_{y}.n_{y}}{\widehat{F}_{t}^{(i)}} 		&= \widehat{C}^{(i)} P_{t|t-1}^{(i)} \widehat{C}^{(i)'} + \widetilde{H}^{(i)} \\
\underset{n_{x}.1}{\widehat{X}_{t|t}^{(i)}} 		&= \widehat{X}_{t|t-1}^{(i)} + \widehat{P}_{t|t-1}^{(i)} \widehat{C}^{(i)} \left(\widehat{F}_{t}^{(i)}\right)^{-1} \widehat{v}_{t}^{(i)}\\	
\underset{n_{x}.1}{\widehat{X}_{t+1|t}^{(i)}} 		&= \widehat{A}^{(i)} \widehat{X}_{t|t}^{(i)}\\
\underset{n_{x}.n_{x}}{\widehat{P}_{t|t}^{(i)}}		&= \widehat{P}_{t|t-1}^{(i)} - \widehat{P}_{t|t-1}^{(i)} \widehat{C}_{t}^{(i)'} \left(\widehat{F}_{t}^{(i)}\right)^{-1} \widehat{C}^{(i)} \widehat{P}_{t|t-1}^{(i)}\\
\underset{n_{x}.n_{x}}{\widehat{P}_{t+1|t}^{(i)}}	&= \widehat{A} \widehat{P}_{t|t}^{(i)} \widehat{A}' + \widetilde{Q}
\end{aligned}
\end{equation*}



We deal with missing values by following \cite{DurbinKoopman2012} for how to deal with missing values.

\subsection{Log-likelihood}


\subsection{Kalman Smoother}

%-----------------------------------------%
\section{Estimation}

Following ... we use a two stage-procedure for the estimation, by first estimating the factors using principle components on a subset of the dataset, with a balanced panel and then using projections get the initial parameters of the model.

Then in the second stage we using the EM algorithm and quasi-maximum likelihood to estimate the model.

The basic steps of the estimation procedure is:
\begin{enumerate}
	\item 	PCA estimates of initial states and parameters
	\item 	Set up the modle with the initial parameter and starting values
	\item 	Compute an estimate of the unobserved states (conditional of the parameters)
	\item 	Re-estimate the parameters of the model (conditional of the estimates of the states)
	\item 	Repeat step (3) and (4) until convergence -- can add additional numerical maximisation step after EM-algorithm has converged to check if it has truly converged.
\end{enumerate}

%--------------------------%
\subsection{Stage (1) Initial conditions and parameters}

WE follow \cite{DozGiannoneReichlin2011} in using a two-stage procedure to estimate the model and augment the last stage with an additional update of the log-likelihood.
Using PCA we estimate the factors on a balanced panel of the data, without any missing values.

The balanced panel is chosen from the monthly variable with the following standardisation:

\begin{equation*}
\left(\frac{y_{t,i}^{m} - \mu_{i}^{m}}{\sigma_{i}^{m}}\right) \quad \forall i=1,\dots,n_{m}
\end{equation*}

Using principal components
And the initial parameters are then estimated on the demeaned variables:

\begin{equation}
\begin{aligned}
\tilde{y}_{t}^{m}  	&= y_{t}^{m} - \mu_{m} &&= \lambda_{Q} f_{t} + u_{t}^{m}, \quad &&u_{t}^{m} &= \Theta^{m}(L)u_{t} + \varepsilon_{t}^{m},\quad \varepsilon_{t}^{m}\sim N(\textbf{0}_{n_{m}}, H^{m}) \\
\tilde{y}_{t}^{Q}	&= y_{t}^{Q} - \mu_{Q} &&= \lambda_{Q} f_{t} + u_{t}^{Q}, \quad &&u_{t}^{Q} &= \Theta^{Q}(L)u_{t} + \varepsilon_{t}^{Q},\quad \varepsilon_{t}^{Q} \sim N(\textbf{0}_{n_{Q}}, H^{Q})
\end{aligned}
\end{equation}

And using the OLS projection to get the initial parameters for the monthly variables:
\begin{equation*}
\widehat{\lambda}_{m}^{(0)} = \widetilde{y}^{m} \widehat{\textbf{f}}_{PCA} \left(\widehat{\textbf{f}}_{PCA}'\widehat{\textbf{f}}_{PCA}\right)^{-1}
\end{equation*}

%--------------------------%
\subsection{Stage (2) Estimation: Maximum Likelihood}

Outlining the EM-algorithm for estimating the Maximum Likelihood parameters of the model.

\subsection{Step (2.1) Quasi-difference of the series}
Quasi-difference of the series conditional on parameters from iteration $i$
\begin{equation*}
\underset{n_{y}.1}{\widetilde{y}_{t}^{(i)}} =
\begin{bmatrix}
\left(I_{n_{m}} - \Theta_{m}^{(i-1)}(L)\right)\left(y_{t}^{(m)} - \mu_{m}\right)\\
y_{t}^{(q)} - \mu_{q}
\end{bmatrix}
\end{equation*}

%-------------------------------------%	
\subsection{Step (2.2) Re-estimate the factors ($X_{t}^{(i)}$)}
Estimate states (factors) for iteration $i$ conditional on the parameters from iteration $i$

\begin{equation*}
X^{(i)}_{t}(\widetilde{y}_{t}^{(i)})
\end{equation*}

%------------------%	
\subsection{Step (2.3) Re-estimate the parameters ($i+1$)}
Conditional on the states from iteration $i+1$ re-estimate the parameters

The observations equation for the monthly variables (\textbf{OBS: Have to modify}):

\begin{equation}
vec{(\lambda_{m}^{(i+1)})} = 
\left(\sum_{t=1}^{T} f_{t}^{(i)} {f_{t}^{(i)}}' \otimes W_{t}^{m} \right)^{-1}
vec\left(\sum_{t=1}^{T} W_{t}^{m} y_{t}^{(m)*} {f_{t}^{(i)}}'\right)
\end{equation}

Where $i$ refers to the iterations of the EM algorithm.

The updates for the parameters of the quarterly observations are similarly given by:

\begin{equation}
vec{(\lambda_{q}^{(i+1)})} = 
\left(\sum_{t=1}^{T} \widetilde{f}_{t}^{(i)} \widetilde{f}_{t}^{(i)\intercal} \otimes W_{t}^{q} \right)^{-1}
vec\left(\sum_{t=1}^{T} W_{t}^{q} y_{t}^{(q)*} \widetilde{f}_{t}^{(i)\intercal}\right)
\end{equation}

For the transition dynamics of the unobserved factors:
\begin{equation}
\widehat{\Phi}^{(i+1)} =  \sum_{t=1}^{T}\left(f_{t}^{(i)} F_{t-1}^{(i)'}\right) \sum_{t=1}^{T}\left(F_{t-1}F_{t-1}'\right)
\end{equation}

Where $F_{t}$ is given by:
\begin{equation}
F_{t} = 
\begin{bmatrix}
f_{t} & f_{t-1} & \dots & f_{t-p_{f}}
\end{bmatrix}
\end{equation}

The residuals are then estimated as:

\begin{equation*}
\widehat{\eta}_{t}^{(i+1)} = f_{t}^{(i)} - \widehat{\Phi}^{(i+1)} F_{t-1}^{(i)}
\end{equation*}

Which means we can update the estimate of the variance-covariance matrix as (adjusting for the degree of freedoms):

\begin{equation*}
\widehat{Q}^{(i+1)} = \frac{1}{T - p_{f} r}\sum_{t=1}^{T} \left(\eta_{t}^{(i+1)}\eta_{t}^{(i+1)'}\right)
\end{equation*}


The parameters for the moving average component of the quarterly idiosyncratic part can be estimated as:
\begin{equation*}
\widehat{\Theta}_{Q,j}^{(i+1)} = 
\sum_{t=1}^{T} \left(u_{t,j}^{Q}U_{t-1,j}^{Q'}\right)
\sum_{t=1}^{T} \left(U_{t-1,j}^{Q} U_{t-1,j}^{Q'}\right)^{-1}, 
\quad \forall j = 1, \dots, n_{Q}
\end{equation*}

Where the $U_{t-1,j}^{Q}$ variables is defined as:
\begin{equation*}
U_{t-1,j}^{Q} =
\begin{bmatrix}
u_{t-1,j}^{Q} & \dots & u_{t-p_{u},j}^{Q}
\end{bmatrix}
\end{equation*}

The residuals can then be estimated as:
\begin{equation*}
\varepsilon_{t,j}^{Q,(i)} = \widehat{u}_{t,j}^{Q,(i)} - \widehat{\Theta}_{Q,j}^{(i+1)} \widehat{U}_{t,j}^{Q,(i)}
\end{equation*}

And the individual elements of the diagonal variance-covariance matrix can then be estimated as:
\begin{equation*}
\sigma_{Q,j}^{(i+1)} = 
\frac{1}{T - 1}\sum_{t=1}^{T}\left(	\varepsilon_{t,j}^{Q,(i)} 	\varepsilon_{t,j}^{Q,(i)'} \right),
\quad \forall j = 1,\dots,n_{Q}
\end{equation*}

%------------------%
\subsection{Step (2.4) Check convergence}	
Check convergence using the following criteria:
\begin{equation*}
\bar{L} = \frac{L-L}{L}
\end{equation*}





%-----------------------------%
\subsection{Stage (3) Numerical optimisation of log--lik}

Using Newtons methods we re-optimise the parameters and states conditional on the previous EM algorithm solution.

\end{appendices}
%----------------------------------------------------%

\end{document}
